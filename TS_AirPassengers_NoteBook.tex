\PassOptionsToPackage{unicode=true}{hyperref} % options for packages loaded elsewhere
\PassOptionsToPackage{hyphens}{url}
%
\documentclass[]{article}
\usepackage{lmodern}
\usepackage{amssymb,amsmath}
\usepackage{ifxetex,ifluatex}
\usepackage{fixltx2e} % provides \textsubscript
\ifnum 0\ifxetex 1\fi\ifluatex 1\fi=0 % if pdftex
  \usepackage[T1]{fontenc}
  \usepackage[utf8]{inputenc}
  \usepackage{textcomp} % provides euro and other symbols
\else % if luatex or xelatex
  \usepackage{unicode-math}
  \defaultfontfeatures{Ligatures=TeX,Scale=MatchLowercase}
\fi
% use upquote if available, for straight quotes in verbatim environments
\IfFileExists{upquote.sty}{\usepackage{upquote}}{}
% use microtype if available
\IfFileExists{microtype.sty}{%
\usepackage[]{microtype}
\UseMicrotypeSet[protrusion]{basicmath} % disable protrusion for tt fonts
}{}
\IfFileExists{parskip.sty}{%
\usepackage{parskip}
}{% else
\setlength{\parindent}{0pt}
\setlength{\parskip}{6pt plus 2pt minus 1pt}
}
\usepackage{hyperref}
\hypersetup{
            pdftitle={TimeSeries\_AirPassengers},
            pdfborder={0 0 0},
            breaklinks=true}
\urlstyle{same}  % don't use monospace font for urls
\usepackage[margin=1in]{geometry}
\usepackage{color}
\usepackage{fancyvrb}
\newcommand{\VerbBar}{|}
\newcommand{\VERB}{\Verb[commandchars=\\\{\}]}
\DefineVerbatimEnvironment{Highlighting}{Verbatim}{commandchars=\\\{\}}
% Add ',fontsize=\small' for more characters per line
\usepackage{framed}
\definecolor{shadecolor}{RGB}{248,248,248}
\newenvironment{Shaded}{\begin{snugshade}}{\end{snugshade}}
\newcommand{\AlertTok}[1]{\textcolor[rgb]{0.94,0.16,0.16}{#1}}
\newcommand{\AnnotationTok}[1]{\textcolor[rgb]{0.56,0.35,0.01}{\textbf{\textit{#1}}}}
\newcommand{\AttributeTok}[1]{\textcolor[rgb]{0.77,0.63,0.00}{#1}}
\newcommand{\BaseNTok}[1]{\textcolor[rgb]{0.00,0.00,0.81}{#1}}
\newcommand{\BuiltInTok}[1]{#1}
\newcommand{\CharTok}[1]{\textcolor[rgb]{0.31,0.60,0.02}{#1}}
\newcommand{\CommentTok}[1]{\textcolor[rgb]{0.56,0.35,0.01}{\textit{#1}}}
\newcommand{\CommentVarTok}[1]{\textcolor[rgb]{0.56,0.35,0.01}{\textbf{\textit{#1}}}}
\newcommand{\ConstantTok}[1]{\textcolor[rgb]{0.00,0.00,0.00}{#1}}
\newcommand{\ControlFlowTok}[1]{\textcolor[rgb]{0.13,0.29,0.53}{\textbf{#1}}}
\newcommand{\DataTypeTok}[1]{\textcolor[rgb]{0.13,0.29,0.53}{#1}}
\newcommand{\DecValTok}[1]{\textcolor[rgb]{0.00,0.00,0.81}{#1}}
\newcommand{\DocumentationTok}[1]{\textcolor[rgb]{0.56,0.35,0.01}{\textbf{\textit{#1}}}}
\newcommand{\ErrorTok}[1]{\textcolor[rgb]{0.64,0.00,0.00}{\textbf{#1}}}
\newcommand{\ExtensionTok}[1]{#1}
\newcommand{\FloatTok}[1]{\textcolor[rgb]{0.00,0.00,0.81}{#1}}
\newcommand{\FunctionTok}[1]{\textcolor[rgb]{0.00,0.00,0.00}{#1}}
\newcommand{\ImportTok}[1]{#1}
\newcommand{\InformationTok}[1]{\textcolor[rgb]{0.56,0.35,0.01}{\textbf{\textit{#1}}}}
\newcommand{\KeywordTok}[1]{\textcolor[rgb]{0.13,0.29,0.53}{\textbf{#1}}}
\newcommand{\NormalTok}[1]{#1}
\newcommand{\OperatorTok}[1]{\textcolor[rgb]{0.81,0.36,0.00}{\textbf{#1}}}
\newcommand{\OtherTok}[1]{\textcolor[rgb]{0.56,0.35,0.01}{#1}}
\newcommand{\PreprocessorTok}[1]{\textcolor[rgb]{0.56,0.35,0.01}{\textit{#1}}}
\newcommand{\RegionMarkerTok}[1]{#1}
\newcommand{\SpecialCharTok}[1]{\textcolor[rgb]{0.00,0.00,0.00}{#1}}
\newcommand{\SpecialStringTok}[1]{\textcolor[rgb]{0.31,0.60,0.02}{#1}}
\newcommand{\StringTok}[1]{\textcolor[rgb]{0.31,0.60,0.02}{#1}}
\newcommand{\VariableTok}[1]{\textcolor[rgb]{0.00,0.00,0.00}{#1}}
\newcommand{\VerbatimStringTok}[1]{\textcolor[rgb]{0.31,0.60,0.02}{#1}}
\newcommand{\WarningTok}[1]{\textcolor[rgb]{0.56,0.35,0.01}{\textbf{\textit{#1}}}}
\usepackage{graphicx,grffile}
\makeatletter
\def\maxwidth{\ifdim\Gin@nat@width>\linewidth\linewidth\else\Gin@nat@width\fi}
\def\maxheight{\ifdim\Gin@nat@height>\textheight\textheight\else\Gin@nat@height\fi}
\makeatother
% Scale images if necessary, so that they will not overflow the page
% margins by default, and it is still possible to overwrite the defaults
% using explicit options in \includegraphics[width, height, ...]{}
\setkeys{Gin}{width=\maxwidth,height=\maxheight,keepaspectratio}
\setlength{\emergencystretch}{3em}  % prevent overfull lines
\providecommand{\tightlist}{%
  \setlength{\itemsep}{0pt}\setlength{\parskip}{0pt}}
\setcounter{secnumdepth}{0}
% Redefines (sub)paragraphs to behave more like sections
\ifx\paragraph\undefined\else
\let\oldparagraph\paragraph
\renewcommand{\paragraph}[1]{\oldparagraph{#1}\mbox{}}
\fi
\ifx\subparagraph\undefined\else
\let\oldsubparagraph\subparagraph
\renewcommand{\subparagraph}[1]{\oldsubparagraph{#1}\mbox{}}
\fi

% set default figure placement to htbp
\makeatletter
\def\fps@figure{htbp}
\makeatother


\title{TimeSeries\_AirPassengers}
\author{}
\date{\vspace{-2.5em}}

\begin{document}
\maketitle

\begin{Shaded}
\begin{Highlighting}[]
\KeywordTok{library}\NormalTok{(ggfortify)}
\end{Highlighting}
\end{Shaded}

\begin{verbatim}
## Loading required package: ggplot2
\end{verbatim}

\begin{Shaded}
\begin{Highlighting}[]
\KeywordTok{library}\NormalTok{(tseries)}
\KeywordTok{library}\NormalTok{(forecast)}
\KeywordTok{data}\NormalTok{(AirPassengers)}
\NormalTok{ts_AirPassengers <-}\StringTok{ }\NormalTok{AirPassengers}
\KeywordTok{class}\NormalTok{(ts_AirPassengers);}\KeywordTok{head}\NormalTok{(ts_AirPassengers);}\KeywordTok{tail}\NormalTok{(ts_AirPassengers);}\KeywordTok{dim}\NormalTok{(ts_AirPassengers)}
\end{Highlighting}
\end{Shaded}

\begin{verbatim}
## [1] "ts"
\end{verbatim}

\begin{verbatim}
##      Jan Feb Mar Apr May Jun
## 1949 112 118 132 129 121 135
\end{verbatim}

\begin{verbatim}
##      Jul Aug Sep Oct Nov Dec
## 1960 622 606 508 461 390 432
\end{verbatim}

\begin{verbatim}
## NULL
\end{verbatim}

\begin{Shaded}
\begin{Highlighting}[]
\CommentTok{# Dimensions = NULL ??}
\CommentTok{# Check for Missing or NA }
\KeywordTok{sum}\NormalTok{(}\KeywordTok{is.na}\NormalTok{(ts_AirPassengers)) }\CommentTok{# No Missing values - No NA }
\end{Highlighting}
\end{Shaded}

\begin{verbatim}
## [1] 0
\end{verbatim}

\begin{Shaded}
\begin{Highlighting}[]
\CommentTok{# }
\end{Highlighting}
\end{Shaded}

\begin{Shaded}
\begin{Highlighting}[]
\CommentTok{# Check Frequency of TimeSeries and the Cyclic part of the TS }
\KeywordTok{frequency}\NormalTok{(ts_AirPassengers);}\KeywordTok{cycle}\NormalTok{(ts_AirPassengers)}
\end{Highlighting}
\end{Shaded}

\begin{verbatim}
## [1] 12
\end{verbatim}

\begin{verbatim}
##      Jan Feb Mar Apr May Jun Jul Aug Sep Oct Nov Dec
## 1949   1   2   3   4   5   6   7   8   9  10  11  12
## 1950   1   2   3   4   5   6   7   8   9  10  11  12
## 1951   1   2   3   4   5   6   7   8   9  10  11  12
## 1952   1   2   3   4   5   6   7   8   9  10  11  12
## 1953   1   2   3   4   5   6   7   8   9  10  11  12
## 1954   1   2   3   4   5   6   7   8   9  10  11  12
## 1955   1   2   3   4   5   6   7   8   9  10  11  12
## 1956   1   2   3   4   5   6   7   8   9  10  11  12
## 1957   1   2   3   4   5   6   7   8   9  10  11  12
## 1958   1   2   3   4   5   6   7   8   9  10  11  12
## 1959   1   2   3   4   5   6   7   8   9  10  11  12
## 1960   1   2   3   4   5   6   7   8   9  10  11  12
\end{verbatim}

\begin{Shaded}
\begin{Highlighting}[]
\CommentTok{# Check the summary of TS data }
\KeywordTok{summary}\NormalTok{(ts_AirPassengers)}
\end{Highlighting}
\end{Shaded}

\begin{verbatim}
##    Min. 1st Qu.  Median    Mean 3rd Qu.    Max. 
##   104.0   180.0   265.5   280.3   360.5   622.0
\end{verbatim}

\begin{Shaded}
\begin{Highlighting}[]
\CommentTok{# Plot Raw TimeSeries using the inbuilt base PLOT }
\KeywordTok{plot}\NormalTok{(ts_AirPassengers,}\DataTypeTok{xlab=}\StringTok{"Months"}\NormalTok{, }\DataTypeTok{ylab =} \StringTok{"Air Passengers RAW Data"}\NormalTok{,}\DataTypeTok{main=}\NormalTok{(}\StringTok{"data(AirPassengers) Raw Data TimeSries Plot"}\NormalTok{))}
\end{Highlighting}
\end{Shaded}

\includegraphics{TS_AirPassengers_NoteBook_files/figure-latex/unnamed-chunk-2-1.pdf}

\begin{Shaded}
\begin{Highlighting}[]
\KeywordTok{boxplot}\NormalTok{(ts_AirPassengers}\OperatorTok{~}\KeywordTok{cycle}\NormalTok{(ts_AirPassengers),}\DataTypeTok{xlab=}\StringTok{"Months"}\NormalTok{, }\DataTypeTok{ylab =} \StringTok{"Air Passengers Count "}\NormalTok{ ,}\DataTypeTok{main =}\StringTok{"Seasonality Plot - Air Passengers "}\NormalTok{)}
\end{Highlighting}
\end{Shaded}

\includegraphics{TS_AirPassengers_NoteBook_files/figure-latex/unnamed-chunk-3-1.pdf}

\begin{Shaded}
\begin{Highlighting}[]
\CommentTok{#Seasonality Plot − Air Passengers}
\end{Highlighting}
\end{Shaded}

\end{document}
